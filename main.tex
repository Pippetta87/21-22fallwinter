\documentclass[10pt,xcolor={usenames},fleqn,mathserif,serif]{beamer}
%%%Usefull link
%tikz-equations:
%http://www.wekaleamstudios.co.uk/posts/creating-a-presentation-with-latex-beamer-equations-and-tikz/
\hypersetup{pdfpagemode=FullScreen}
%% colors
\definecolor{bittersweet}{rgb}{1.0, 0.44, 0.37}
\definecolor{brilliantlavender}{rgb}{0.96, 0.73, 1.0}
\definecolor{antiquefuchsia}{rgb}{0.57, 0.36, 0.51}
\definecolor{violetw}{rgb}{0.93, 0.51, 0.93}
\definecolor{Veronica}{rgb}{0.63, 0.36, 0.94}
\definecolor{atomictangerine}{rgb}{1.0, 0.6, 0.4}
\definecolor{darkgray}{rgb}{0.66, 0.66, 0.66}
\definecolor{brightcerulean}{rgb}{0.11, 0.67, 0.84}
\definecolor{cadmiumorange}{rgb}{0.93, 0.53, 0.18}
\definecolor{ochre}{rgb}{0.8, 0.47, 0.13}
\definecolor{midnightblue}{rgb}{0.1, 0.1, 0.44}
\definecolor{lemon}{rgb}{1.0, 0.97, 0.0}
\definecolor{grey}{rgb}{0.7, 0.75, 0.71}
\definecolor{amber}{rgb}{1.0, 0.75, 0.0}
\definecolor{almond}{rgb}{0.94, 0.87, 0.8}
\definecolor{bf}{RGB}{88, 86, 88}
\definecolor{bb}{RGB}{177, 177, 177}
%%%%%%%%%%%%%%%%%%%%%%%%%%%%%%%%%%% importa pacchetti
\usepackage{usepkg}
%%%%%%%%%%%%%%%%%%%%%%%%%%%%%%%%%%% Funzioni generali
\usepackage{functions}
%http://tex.stackexchange.com/questions/246/when-should-i-use-input-vs-include
\newcommand{\setmuskip}[2]{#1=#2\relax} %%problem usinig mu with calc (req by mathtools) loaded
\usepackage{sources}
%\usepackage{length}
%%%%%%%%%%%%%%%%%%%%%%%%%%%%%%%%%%% Funzioni per questo file main
\usepackage{mathOp}
\usepackage{beamersetup}
\def\status{coazione}
\def\keeptrying{coazione}
\usepackage{LocalF}
%%%%%%%%%%%%%%%%%%%%%%%%%%%%%%%%%
\title{Autunno-Inverno 20/21}
% A subtitle is optional and this may be deleted
\subtitle{Multi-Wavelenght detectors and their calibration, radiation transfer in ISM, and what ever come on us (Teorie gravitazione, condensed matter e dispositivi fotonici}
\date{Lavori a Ruffolo, lezioni, \today}
% - Either use conference name or its abbreviation.
% - Not really informative to the audience, more for people (including
%   yourself) who are reading the slides online
% Let's get started
\begin{document}

\addtobeamertemplate{block begin}{\setlength\abovedisplayskip{2pt}\setlength\belowdisplayskip{2pt}\setlength\abovedisplayshortskip{2pt}\setlength\belowdisplayshortskip{2pt}}

\addtobeamertemplate{block begin}{\vspace*{-3pt}}{}
\addtobeamertemplate{block end}{}{\vspace*{-3pt}}

\begin{frame}
  \titlepage
\end{frame}

\begin{frame}{tempo interno per}
\tableofcontents[onlyparts]
\end{frame}

% Section and subsections will appear in the presentation overview
% and table of contents.
%\frame{\tableofcontents[onlyparts]}
%\begin{frame}{Argomenti}
%  \tableofcontents[part=1,hideallsubsections%,pausesections
%  ]
%  % You might wish to add the option [pausesections]
%\end{frame}

\part{Organizzazione}\linkdest{organization}

\begin{frame}{granular: stellare ottobre - }
    \begin{itemize}
        \item tempo: 10.30-13, 15-19, 23.30-3.30
        \item asd-ripetere: testgof, cramer-rao, likelihood for binned data and extended likelihood
        \item workout-compiti(fare cose che abbiano senso): 30min/exercise exams with sol/45min/per exercise exams without sols
        \item finire rad transf: mihalas:?? (2.00h)
        \item home: bagni, piatti, iola vicino susino per basilico/prezzemolo/??
        \item impianto: campanello (0.30h), camera(1.30h), luce 3 vie(0.30h), luci sala (1.30h), cambio fili sala/soffitta(2.00h), ripetitore wifi in terrazza?
        \item workout: bike(mar 19.30-21, mer 19.30-21)
    \end{itemize}
\end{frame}

\begin{frame}[allowframebreaks]{weekly}
Lentezza, fatica, lontananza della concretezza. 
\begin{itemize}
    \item Lun - 10-11: workout/doccia; 11-14: criticit\'a; 14.00-15.00: pranzo; 15-19: asd exams; 19-22: impianto/orto, 22-0.30: piatti/cena 0.30-05: stellare/mihalas
    \item Mar - 10-11: sveglia/workout/abluz, 11-12: telefono numero emergenza sara/criticit\'a asd, 12-12.30: preparo vaccino e reso amazon 12.30-14.30: vaccino/reso amazon/(pam?), 15.00-16.00: pranzo, 16.00-17.00: criticit\'a asd, 17.30-19.30: giardinaggio con mamma, 19.30-21.00: criticit\'a asd, 21.30-22.30: workout punchbag, 23.00-0.00: doccia/cena, 0.00-2.30: stellare/mihalas, 02.30-5.30 asd exams
    \item Mer - 10.30-11.30: workout/doccia/colazione, 11.30.13.30: criticit\'a asd, 14.00-15.00: scatole per soffitta/ballino calce e gesso, 15-16: lavanderia/parrucchiere all'arbia, 16-17: pranzo, 17-19: criticit\'a asd, 18.30-20: workout bike, 20.30-22.30: asd exams 22.30-01: mihalas/stellare, 01.00-04.00: criticit\'a asd
    \item Gio - 10.40-11.40: abluzioni, 11.40-14.30: lavoro soffitta/lavo macchina, 14.30-15.30: pranzo, 16.00-17.00: lavanderia/taglio capelli? 17.00-18: criticit\'a asd, 18.00-22.00: impianto luci/prese cantina, citofono, rimonto bike, 22-23: doccia/cena, 23-2.30: asd criticit\'a, 2.30-5: asd exams
    \item Ven. 11-12: lavo macchina; 12-14.00: asciugo macchina/citofono/bike/rimonto macchina/doccia; 14-15: pranzo/versamento BNL (14.50-16.05); 15-16: coperta lana/parrucchiiere carmen(15.30); 16.30-18.30: asd criticit\'a; 18.30-19.30: rimonto bike/workout bike, 20-21: cena; 21-22.30: panni per 7/8 gg; 22.30-0: carico macchina
    \item Sab - 10.30-11.30: abdominal workout/colazione, 11.30-13.30: criticit\'a asd, 13.30-14.00: ordin cose da studiare/garage; 14.15-17.45: pranzo, ordini amazon, etc 18.30-20.30: workout bike, 20.30-21.30: criticit\'a asd 21.30-22.30: cena, 22.30-1.30: stellare/mihalas, 1.30-4.00: asd exams, 4-5: organizzo
    \item Dom - 9.30-10.30: workout/colazione 10.30-12.00: scatola soffita: interrompo corrente, taglio corrugato, estrazione fili, buco scatola, gesso!, ricollego fili e tiro fibra di pi\'u in terrazza, 12.30-14: finisco di organizzare materiale, 14-15: pranzo, 15.00-17.30: criticit\'a asd, 18-19.30: mamma e il trituratore/garage, 19.30-20.30: workout bike, 21-22: criticit\'a asd, 22-23: cena, 23-01: criticit\'a asd, 01-04: asd exams
\end{itemize}
\end{frame}

\begin{frame}[allowframebreaks,label={byday}]{Emploi du temp (by day)}
\begin{itemize}
\item Lun. 8.30-11.15: computing method (int. rad. mat.); 10.30-12.15: metod. sper. astro; 14.30-16.15: IF; 16.30-18.15: Fis. neutrini (stato solido/plasmi)
\item Mar. 9-10.15: plasmi/8.30-10.15: rel. gen.; 10.30-12.15: plasmi; 12.30-14.15: proc astro (IF); 14.30-15.30: Fis. stat.; 15.30-17.30: fis. neutrini;
\item Mer. 10.30-12.15: fis. stat.; 16.30-18.15: rel. gen.
\item Gio. 8.30-11.15: comp. method; 10.30-12.15: solid state (fis. nucl.); 14.30-15.15: met. sper. astrop.; 15.30-16.30: rel. gen.; 18.30-19.30: int. rad mat./16.30-19.15: proc. astro
\item Ven. 10.30-12.15: int. rad. mat. (IF); 12.30-14.15: plasmi; 14.30-16.15: fis. stat. (met. sper. astrop.);
\item Sab. 9.30-12.00: nucl.
\end{itemize}
\end{frame}

\begin{frame}[allowframebreaks,label={lm2122I}]{Emploi du temp}
\begin{itemize}
\item Analisi statistica dei dati: 9, PUNZI, mar 10:30‑12:15 Aula Fib O, mer 18:30‑19:30 Aula Fib G1, ven 8:30‑10:15 Aula Fib O
\item Computing methods for experimental physics and data analysis: 9, RIZZI, Lun 8:30‑11:15 Aula Fib O, Gio 8:30‑11:15 Aula Fib O
\item Cosmologia del primo universo: 9, GRASSO, Lun 10:30‑12:15 Aula Fib V1, Mer 14:30‑16:15 Aula Fib B, Gio 8:30‑10:15 Aula Fib G1
\item Fisica Statistica: 9, ROSSINI, Mar 14:30‑15:30 Aula Fib B1, Mer 10:30‑12:15 Aula Fib V1, Ven 14:30‑16:15 Aula Fib G1
\item Fisica dei plasmi	9, CALIFANO, Lun 16:30‑18:15 Aula Fib O, Mar 9:00‑10:15 Aula Fib G1, Ven 12:30‑14:15 Aula Fib G1
\item Fisica dello stato solido	9, RODDARO, Lun 16:30‑18:15 Aula Fib B, Mar 10:30‑12:15 Aula Fib V1, Gio 10:30‑12:15 Aula Fib V1
\item Fisica nucleare, 9 BOMBACI, Gio 10:30‑12:15 Aula Fib T1, Sab 9:30‑12:00 Aula Fib G1
\item Fondamenti di interazione radiazione materia, 9, TREDICUCCI, Lun 8:30‑10:15 Aula Fib V1, Gio 18:30‑19:30 Aula Fib V1, Ven 10:30‑12:15 Aula Fib V1
\item Interazioni fondamentali,9, FORTI, Lun 14:30‑16:15 Aula Fib O, Mar 12:30‑14:15 Aula Fib O, Ven 10:30‑12:15 Aula Fib O
\item Introduzione alla fisica dei neutrini, 3, CAVASINNI, Mar 16:30‑18:15 Aula Fib T1
\item Metodologie sperimentali per la fisica delle astroparticelle, 9, ROSA POGGIANI, Lun 10:30‑12:15 Aula Fib T1, Gio 14:30‑15:15 Aula Fib T1, Ven 14:30‑16:15 Aula Fib S1
\item Processi astrofisici, 9, DEL POZZO, Mar 12:30‑14:15 Aula Fib T1, Gio 16:30‑19:15 Aula Fib T1
\item Relativita generale, 9, BOLOGNESI, Mar 8:30‑10:15 Aula Fib B1, Mer 16:30‑18:15 Aula Fib B, Gio 15:30‑16:30 Aula Fib B
\end{itemize}
\end{frame}

\section{AO}

\begin{itemize}
    \item Visibility exercise
    \item star field exercise
    \item spectrum analysis
\end{itemize}

\section{ISM}

\begin{itemize}
    \item How we know there's a ISM? (19/02)
    \item What is a gas? Condition for using kinetic approx? What ''pressure'' mean? What ''moving'' means? (19/02)
    \item ISM absorption: clouds - IGM absorption: Ly forest (26/02)
    \item Curve of Growth (26/02):
    \item Exercise: Curve of Growth for? (26/02)
    \item Linewidth, turbulence and element abundances (01/03)
    \item 
\end{itemize}


\section{asd}

\begin{itemize}
    \item ripetere/registrare
    \item compiti
    \item succo/deeper
\end{itemize}

\subsection{Compiti}

\begin{itemize}
    \item 06/04 - 10-11.30: passaggio a livello (17/09/20), 22-23.30: membrane cellulari (10/02/20)
    \item 07/04 - 10-11.30: generazione 2 numeri casuali indipendenti (21/11/19), 22-23.30: (09/09/19) 1) $x_1, x_2\in U[0,1]$, $x=$\lbt{x_2:\ x_1>f}{mx_2}. 2) Misura massa m con contributo f di nuova fisica
    \item 08/04 - 10-11.30: (08/01/20) $x_1,x_2\in U[-m/2,m/2]$ $y=|x_1+x_2|$ 22-23.30: (01/02/18) 1) Prove di esame a risposta multipla 2) Test per dado truccato
    \item 09/04 - 10-11.30: (16/07/19) generazione numeri pseudo-casuali 22-23.30 (28/05/19) 1) $p(x;\theta)=\frac{\theta3^{\theta}}{x^{\theta+1}}$ 2) probabilit\'a genotipi AA, Aa, aa
    \item 10/04 10-11.30 (11/02/19) 1) Distribuzione esponenziale 2) Contachilometri senza decimale 22-23.30 (15/01/21) Sistema di acquisizione con pre-scale 
    \item 11/04 10-11.30 (08/02/21) Rivelatore vita media di particelle cariche pi\'u rumore
\end{itemize}

\subsection{Ripetere/registrare/succo/deeper}

\begin{itemize}
    \item 
\end{itemize}

\section{ISM}

\subsection{Program}

\begin{itemize}
    \item Physical state of medium (microscale). Heliosphere. ISM (Atomic, Molecular, Dust), IGM (Atomic, Dust): thermal, non-thermal.
    \item Dynamic state. Flows: Large scale (\SIrange{e2}{e3}{\parsec}), Small scale ($<\SI{1}{\parsec}$), kinematics.
    \item Constitution/Structure of the Medium
    \item Energetics Imbalances
\end{itemize}

\begin{itemize}
    \item Exercise: Trans-Ocean cable. Heavyside problem: How quickly can I press the button without having signal smeared-out.
    Assuming: plasma of ISM not optically thin, index of refraction calculation, behaviour (fluid-like (MHD), gas+grain+charges)
    \item 
\end{itemize}

\end{document}